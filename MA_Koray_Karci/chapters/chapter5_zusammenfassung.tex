\chapter{Zusammenfassung und Ausblick}\label{ch:ch4}
Im nächsten Schritt werden zwei verschiedene Online-Parameterschätzverfahren beschrieben und die dazugehörigen Simulationsergebnisse diskutiert. Die Parameter werden dabei in den verschieden Schaltphasen geschätzt.

Des weiteren kann durch die Betrachtung der Widerstandskräfte $F_\mathrm{r}(v)$ und $F_\mathrm{ad}(v)$ als Eingangsgrößen, das System weiterhin in lineare Form
\begin{equation}\label{eq:sys_linex}
\dot{\pmb{x}}_\mathrm{ex} = \pmb{A}_\mathrm{ex}\,\pmb{x}_\mathrm{ex} + \pmb{B}_\mathrm{ex}\,\pmb{u}_\mathrm{ex}
\end{equation}
angegeben werden. Hierfür werden die beiden Widerstandskräfte in \eqref{eq:dynwc} durch $T^*_\mathrm{res}$ ersetzt, sodass sich 
\begin{equation}\label{eq:dynwc_lin}
\dot{\omega}_\mathrm{C} = \left[ 2\,\left(k_\mathrm{ss}\left(\frac{\phi_2}{i_{\mathrm{D}}} - \phi_\mathrm{C}\right) + d_\mathrm{ss}\left(\frac{\omega_2}{i_{\mathrm{D}}} - \omega_\mathrm{C}\right)\right) - T^*_\mathrm{res} - r_\mathrm{dyn}\, F_\mathrm{g}\right]/I^\mathrm{eff}_\mathrm{C}.
\end{equation}
 ergibt. Der Eingangsvektor des linearen Systems ist dann definiert als 
\begin{equation}
\pmb{u}_\mathrm{ex} = \begin{bmatrix} \mathrm{T}_{\mathrm{In}} & T^*_\mathrm{res} & \mathrm{T}_{\mathrm{K81}} & \mathrm{T}_{\mathrm{K38}} & \mathrm{T}_{\mathrm{B08}} & \mathrm{T}_{\mathrm{B05}} & \mathrm{T}_{\mathrm{B06}} & \mathrm{T}_{\mathrm{K27}} & \zeta \end{bmatrix}^T.
\end{equation}

Die Funktion $\pmb{f}(\pmb{x}_\mathrm{ex})$ und die Eingangmatrix $\pmb{B}_\mathrm{ex,nl}$ des nichtlinearen Systems, sowie die Systemmatrix $\pmb{A}_\mathrm{ex}$ und die Eingangmatrix $\pmb{B}_\mathrm{ex}$ des linearen Systems sind im Anhang mit numerischen Werten für die Zähnezahlen der Zahnräder und der Massenträgheitsmomente angegeben.


Mit den eingesetzten numerischen Werten der Zahnräder, der Massenträgheitsmomente, der Übersetzung der Differentials und dem dynamischen Radius der Räder ergeben sich die Systemmatrizen zu
\begin{align}
\pmb{A}_\mathrm{ex,23} &= \begin{bmatrix} 0 & -0,799\,d_\mathrm{ss} & -1,97\,k_\mathrm{ss} & 1,97\,d_\mathrm{ss} \\
0 & -0,248\,d_\mathrm{ss} & -0,612\,k_\mathrm{ss} & 0,612\,d_\mathrm{ss}\\
0 & 0,405 & 0 & -1,0\\
0 & \frac{0,81\,d_\mathrm{ss}}{m_\mathrm{veh}\,0,108 + 6,4} & \frac{2,0\,k_\mathrm{ss}}{m_\mathrm{veh}\,0,108 + 6,4} & \frac{-2,0\,d_\mathrm{ss}}{m_\mathrm{veh}\,0,108 + 6,4})\end{bmatrix}\\
\nonumber \\
\pmb{B}_\mathrm{ex,23} &= \begin{bmatrix} 10,7 & -2,79 & -17,0 & 0 & 0\\
2,44 & 0,0135 & -2,41 & 0 & 0 \\
0 & 0 & 0 & 0 & 0\\
0 & 0 & 0 & \frac{-1}{0,108\,m_\mathrm{veh} + 6,4} & \frac{-3,22\,m_\mathrm{veh}}{0,108\,m_\mathrm{veh} + 6,4} \end{bmatrix}
\end{align}